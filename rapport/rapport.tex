\documentclass[10pt,a4paper]{article}
\usepackage[utf8]{inputenc}
\usepackage[french]{babel}
\usepackage{amsmath}
\usepackage{amsfonts}
\usepackage{amssymb}
\author{Pierre Gimalac \& Alexandre Moine\\\small{Encadré par François Laroussinie}}
\title{Rapport de Projet\\Mathématiques-Informatique}
\begin{document}
\maketitle

\section{CTL, une logique temporelle}
\subsection{Définition}
CTL (pour \textit{Computation Tree Logic}) est une logique temporelle ; CTL contient la logique propositionnelle usuelle et ajoute des opérateurs temporels

Elle permet par exemple d'exprimer des propriétés comme ``il existe une exécution telle que la variable \textit{a} soit toujours vraie''.

\subsection{Grammaire}
La grammaire de CTL est définie de la manière suivante:
\begin{align*}
\phi &::= \bot \mid \top \mid p \mid \neg p \mid \phi\land\phi \mid \phi\lor\phi \mid \\
&\quad \mbox{AX }\phi \mid \mbox{EX }\phi \mid
\mbox{A }\phi \mbox{ W } \phi \mid \mbox{E }\phi \mbox{ W } \phi \mid
\mbox{A }\phi \mbox{ U } \phi \mid \mbox{E }\phi \mbox{ U } \phi
\end{align*}

On peut définir la fonction $neg:CTL \to CTL$:
\begin{align*}
&neg(\top) = \bot \quad neg(\bot) = \top\\
&neg(p) = \neg p \quad neg(\neg p) = p\\
&neg(a \lor b) = neg(a) \land neg(b)\\
&neg(a \land b) = neg(a) \lor neg(b)\\
&neg(\mbox{AX } \phi) = \mbox{EX }neg(\phi)\\
&neg(\mbox{EX } \phi) = \mbox{AX }neg(\phi)\\
&neg(\mbox{E } \phi \mbox{ U } \psi) =\mbox{A } (neg(\psi)) \mbox{ W } (neg(\phi) \land neg(\psi))\\
&neg(\mbox{A } \phi \mbox{ U } \psi) =\mbox{E } (neg(\psi)) \mbox{ W } (neg(\phi) \land neg(\psi))\\
&neg(\mbox{E } \phi \mbox{ W } \psi) =\mbox{A } (neg(\psi)) \mbox{ U } (neg(\phi) \land neg(\psi))\\
&neg(\mbox{A } \phi \mbox{ W } \psi) =\mbox{E } (neg(\psi)) \mbox{ U } (neg(\phi) \land neg(\psi))
\end{align*}

On peut alors définir d'autres opérations :
\begin{itemize}
	\item $\mbox{AF } \phi := \mbox{A } \top \mbox{ U } \phi$
	\item $\mbox{EG } \phi := neg(\mbox{AF } (neg (\phi)))$
	\item $\mbox{EF } \phi := \mbox{E } \top \mbox{ U } \phi$
	\item $\mbox{AG } \phi := neg(\mbox{EF } (neg (\phi)))$
\end{itemize}

\section{Model-checking de CTL: parcours de graphe}

\section{Model-checking de CTL: jeux faibles de parité}
\end{document}
