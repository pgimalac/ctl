\documentclass[10pt,a4paper]{article}
\usepackage[utf8]{inputenc}
\usepackage[french]{babel}
\usepackage{amsmath}
\usepackage{amsfonts}
\usepackage{amssymb}
\author{Pierre Gimalac \& Alexandre Moine\\\small{Encadré par François Laroussinie}}
\title{Rapport de Projet\\Mathématiques-Informatique}
\begin{document}
\maketitle

\section{CTL, une logique temporelle}
\subsection{Définition}
CTL (pour \textit{Computation Tree Logic}) est une logique temporelle ; CTL contient la logique propositionnelle usuelle et ajoute des opérateurs temporels

Elle permet par exemple d'exprimer des propriétés comme ``il existe une exécution telle que la variable \textit{a} soit toujours vraie''.

\subsection{Grammaire}
La grammaire de CTL est définie de la manière suivante :
\begin{align*}
\phi &::= \bot \mid \top \mid p \mid \neg \phi \mid \phi\land\phi \mid \phi\lor\phi \mid \\
&\quad \mbox{AX }\phi \mid \mbox{EX }\phi \mid
\mbox{A }\phi \mbox{ W } \phi \mid \mbox{E }\phi \mbox{ W } \phi \mid
\mbox{A }\phi \mbox{ U } \phi \mid \mbox{E }\phi \mbox{ U } \phi
\end{align*}

Ici, $p$ représente une proposition atomique d'un ensemble $\Omega$.

On définit aussi d'autres opérateurs utiles :
\begin{itemize}
	\item $\mbox{AF } \phi := \mbox{A } \top \mbox{ U } \phi$
	\item $\mbox{EG } \phi := \neg (\mbox{AF } (\neg \phi))$
	\item $\mbox{EF } \phi := \mbox{E } \top \mbox{ U } \phi$
	\item $\mbox{AG } \phi := \neg (\mbox{EF } (\neg \phi))$
\end{itemize}

\subsection{Sémantique de CTL}
Les modèles de CTL sont appelés des \emph{structure de Kripke}.

\subsubsection{Structures de Kripke}
Une structure de Kripke sera ici identifié à la donnée de $(Q,T,q_0,l)$ où:
\begin{itemize}
\item Le couple $(Q,T, q_0)$ est un automate : $Q$ est un ensemble d'états et $T \subseteq Q \times Q$ est l'ensemble des transitions. $q_0 \in Q$ représente l'état de départ.
\item $l : Q \to \Omega$ est une fonction d'étiquetage.
\end{itemize}

\subsubsection{Exécution}
Soit $\mathcal{A} = (Q,T,q_0,l)$ une structure de Kripke.
On dira que $\sigma : \mathbb{N} \to Q$ est une \emph{exécution} de $\cal{A}$ (ne partant pas forcément de l'état initial) si et seulement si $\forall i \in \mathbb{N}, (\sigma (i), \sigma (i+1)) \in T$.

\subsubsection{Satisfaction}
Soit $\mathcal{A} = (Q,T,q_0,l)$ une structure de Kripke, $\phi$ une formule de CTL et $\sigma$ une exécution de $\mathcal{A}$.\\
Pour tout $i \in \mathbb{N}$, on dira que "$\phi$ est vraie au temps $i$ de l'exécution de $\sigma$", et on notera $A,\sigma,i \vDash \phi$ (on omettra souvent la donnée de $\mathcal{A}$ dans notre écriture) si et seulement si, par induction sur $\phi$:\\

\begin{itemize}
	% TODO bot and top
\item $\phi = p$: $p \in l (\sigma(i))$
\item $\phi = \neg \psi$: $\sigma,i \nvDash \psi$
\item $\phi = \psi_1 \lor \psi_2$: $\sigma,i \vDash \psi_1$ ou $\sigma,i \vDash \psi_2$
\item $\phi = \psi_1 \land \psi_2$: $\sigma,i \vDash \psi_1$ et $\sigma,i \vDash \psi_2$
\end{itemize}


\subsection{Négation}
On peut définir la fonction $neg:CTL \to CTL$ qui permet de transformer une formule de CTL de la forme $\neg \phi$ en formule équivalente où toutes les négations sont sur les littéraux :
\begin{align*}
&neg(\top) = \bot \quad neg(\bot) = \top\\
&neg(p) = \neg p \quad neg(\neg \phi) = \phi\\
&neg(a \lor b) = neg(a) \land neg(b)\\
&neg(a \land b) = neg(a) \lor neg(b)\\
&neg(\mbox{AX } \phi) = \mbox{EX }neg(\phi)\\
&neg(\mbox{EX } \phi) = \mbox{AX }neg(\phi)\\
&neg(\mbox{E } \phi \mbox{ U } \psi) =\mbox{A } (neg(\psi)) \mbox{ W } (neg(\phi) \land neg(\psi))\\
&neg(\mbox{A } \phi \mbox{ U } \psi) =\mbox{E } (neg(\psi)) \mbox{ W } (neg(\phi) \land neg(\psi))\\
&neg(\mbox{E } \phi \mbox{ W } \psi) =\mbox{A } (neg(\psi)) \mbox{ U } (neg(\phi) \land neg(\psi))\\
&neg(\mbox{A } \phi \mbox{ W } \psi) =\mbox{E } (neg(\psi)) \mbox{ U } (neg(\phi) \land neg(\psi))
\end{align*}

\section{Model-checking de CTL: parcours de graphe}

\section{Model-checking de CTL: jeux faibles de parité}
\end{document}
